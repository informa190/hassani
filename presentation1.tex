% Valentino Vranic
% Metody inzinierskej prace 2012/13

\documentclass{beamer}

%\usetheme{Warsaw}
%\usetheme{Antibes}
\usetheme{JuanLesPins}
%\usetheme{Goettingen}

\usecolortheme{seahorse}
%\usecolortheme{dolphin}
%\usecolortheme{rose}
% http://deic.uab.es/~iblanes/beamer_gallery/index_by_color.html
%\usecolortheme{beaver}

%\useoutertheme[]{sidebar}

\setbeamercovered{transparent}

\usepackage[english]{babel}
\usepackage[T1]{fontenc}
\usepackage[utf8]{inputenc}
\usepackage{url}

\usepackage{listings}

\lstset{language=C++,basicstyle=\fontsize{8}{9.6}\selectfont,showstringspaces=false,columns=fullflexible,identifierstyle=\ttfamily,keywordstyle=\bfseries,showstringspaces=false,columns=fullflexible}
%\lstset{language=C,basicstyle=\fontsize{10.5}{12.6}\selectfont,identifierstyle=\ttfamily,keywordstyle=\bfseries,showstringspaces=false,columns=fixed}

\def\BibTeX{\textsc{Bib}\kern-.08em\TeX} 

\newcommand{\footcite}[1]{\footnote{\tiny #1}}
\newcommand{\umlet}{.5}
\newcommand{\emp}[1]{\textit{\alert{#1}}}
\newcommand{\kw}[1]{\mbox{\textbf{#1}}}
\newcommand{\id}[1]{\texttt{#1}}
\newcommand{\stl}{\guillemotleft}
\newcommand{\str}{\guillemotright}

\newcommand{\lsti}{\lstinline[basicstyle=\fontsize{10.5}{12.1}\selectfont]}

\newcommand{\ssection}[1]{
	\section{#1}
	\begin{frame}[fragile=singleslide]\frametitle{}
	\Huge #1
	\end{frame}
}

\newcommand{\ssectionn}[1]{
	\section*{#1}
	\begin{frame}[fragile=singleslide]\frametitle{}
	\Huge #1
	\end{frame}
}

\newenvironment{program}{\begin{beamercolorbox}[rounded=true,shadow=true]{block body}\vspace{-4mm}}{\vspace{-2mm}\end{beamercolorbox}}

\setbeamercolor{fvystup}{fg=white,bg=black}
\newenvironment{vystup}{\begin{beamercolorbox}[rounded=true,shadow=true]{fvystup}}{\end{beamercolorbox}}

\newenvironment{poznamka}{\begin{beamercolorbox}[rounded=true,shadow=false]{block body}}{\end{beamercolorbox}}

\setbeamertemplate{footline}[page number]
{
%\insertpagenumber
%\begin{beamercolorbox}{section in head/foot}
%\vskip2pt\insertnavigation{\paperwidth}\vskip2pt
%\end{beamercolorbox}%
}



\author{Nematullah Hasani}
%\url{www.fiit.stuba.sk/~vranic}, \url{vranic@fiit.stuba.sk}}
%{\tiny \url{www.fiit.stuba.sk/~vranic}, \url{vranic@fiit.stuba.sk}}
\institute{
	Institute of Informatics, Information Systems and Software Engineering\\
Faculty of Informatics and Information Technologies\\
Slovak Technical University in Bratislava}

\subtitle{\vspace{3mm} 	Method engineering  2022/2023}

\title{Mobile games
}

\date{\footnotesize 9. november 2022}




\begin{document}

\begin{frame}[fragile=singleslide]
\titlepage
\end{frame}


\begin{frame}[fragile=singleslide]\frametitle{outline}
what is mobile games\\
History of mobile games\\
Types of mobile games\\
Advantage of mobile games\\
Disadvantage of mobile games\\

\end{frame}


\begin{frame}[fragile=singleslide]\frametitle{Mobile games}
\begin{itemize}
    \item Mobile phone has become a part of human life.
    \item Mobile games are not a new phenomenon.
    \item Mobile games are placed in mobile in two ways:
    \begin{itemize}
        \item pre-installed games
        \item Games that are available in Play Store and Gamenet
    \end{itemize}
    \item Every year, many games in different styles are released in the market,
\end{itemize}
\end{frame}

% príkaz \ssection by vytvoril zvláštný slajd s názvom časti - v krátkych prezentáciách to prekáža, lebo oberá o čas

\begin{frame}[fragile=singleslide]\frametitle{اHistory of mobile games}
\begin{itemize}
    \item The history of mobile games does not begin with the introduction of the first electronic game in 1970.
    \item The first mobile game was introduced in 1994 called Tetris
    \item Three years later, Nokia introduced the snake game in 1997
    \item In 2000, the first downloadable game was introduced.
\end{itemize}
\end{frame}




\begin{frame}[fragile=singleslide]\frametitle{Mobile games categories}
\begin{itemize}
\item 
There are different types of mobile games which are categorized according to their main features or objectives. Not based on the type of gameplay they have.
\end{itemize}
\begin{table}[h]
\textbf{Mobile games categories}\\ 
\begin{tabular}{|c |c |}
 \hline\hline
  \textbf{Action games} & \textbf{Action-adventure games}\\ 
  \hline
  \textbf{Adventure games} & \textbf{Strategy games} \\
  
  \hline
 \textbf{Role-playing games} & \textbf{Simulation games} \\
 \hline
 \textbf{Sports games} &\textbf{ Puzzle games}\\ 
 \hline
 \textbf{Idle games} & \\
 \hline
\end{tabular}
\caption{It shows the categories of mobile games.}
\label{table:1}
\end{table}
\end{frame}


\begin{frame}[fragile=singleslide]\frametitle{Advantage of mobile games}
%\includegraphics[scale=.35]{diagram.pdf}
% pridajte vlastný obrázok a zrušte znák % pred príkazom \includegraphics vo formáte PDF prípadne PNG alebo JPG
% scale určuje veľkosť obrázku
\begin{itemize}
    \item Strengthens problem solving skills
    \item It allows for moments to recharge
    \item It increases the efficiency of creativity
    \item It improves morale and team building
    \item t reduces stress
\end{itemize}
\end{frame}


\begin{frame}[fragile=singleslide]\frametitle{Disadvantage of mobile games}
\begin{itemize}
\item Weakness in vision
\item Increasing the probability of disease
\item Reduction of social relations
\end{itemize}
\end{frame}


\end{document}




Text \end{document} za príkazom \end{document} LaTeX ignoruje, takže tu môžete odkladať veci (aj celé slajdy), ktoré nechcete vymazať, lebo ich ešte možno budete potrebovať, avšak ich v danom momente nechcete mať v slajdoch.

